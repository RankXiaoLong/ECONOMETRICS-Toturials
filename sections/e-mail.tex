\chapter{其他}

\section{经济学家预测奥运奖牌}
距离2004年雅典奥运会开幕还有两个多星期。与往常一样,这次体育盛会的奖牌争夺战成为关注焦点。奥运奖牌如何分配不仅是体育界人士争论不休的问题,
经济学家也不甘落后。美国两位经济学家通过分析各国人口,经济总产值以及人均收入等经济因素,预测出本届奥运会的奖牌分配情况,引起广泛兴趣。 
	
\subsection{经济学家无需分析运动员实力}
体育界人士在预测奥运会奖牌分配情况时,往往侧重于各国运动员的身体和心理素质,主要强调运动员的实力。
\textbf{美国达特矛斯学院商学院教授伯纳德和加利福尼亚大学伯克利分校商学院客座教授布斯}则表示,他们无需了解各国运动员的实力,就能预测出本届奥运会的奖牌分配情况。 
	
美国之音报道,这两位经济学家所依靠的主要是经济数据。他们认为,每个国家在奥运会上的表现与其经济实力密切相关,人口和人均收入状况在奥运会奖牌争夺中发挥着极其重要的作用。 
	
两位经济学家在联名发表的论文中说,人口是判断一个国家奥运实力的重要指标,因为一般来说,一个国家的人口越多,它的综合体育实力就应该越强。 
	
国民经济总产值和人均收入状况也很重要,富裕国家可以动员更多的社会资源来发展体育运动,提高在奥运会的竞争水平。当然,除了考察人口和资源等经济因素之外,伯纳德和布斯的研究模式还注重每个国家在以往奥运会上的表现。 
	
历史记录弥补了经济数据分析方面的不足,说明了为什么古巴和俄罗斯这样的国家在人口和经济实力上并不占优势,但是却能经常在奥运会上取得好成绩。 
	
伯纳德和布斯的经济分析模式准确地预测出韩国在2000年悉尼奥运会获得奖牌的数量。对其他各主要国家的奖牌预测误差平均在四枚以内。
	
\subsection{美俄中占据前三位}
两位经济学家对这次雅典奥运会的奖牌分布情况做出如下的预测:美国仍保持奖牌总数第一名,总共将获得九十三枚;其次是俄罗斯,八十三枚;中国五十七枚;德国五十五枚;澳大利亚五十四枚;法国三十七枚;意大利三十三枚。
	
\subsection{奖牌分布趋势}
伯纳德和布斯还预测,自上世纪80年代开始出现的富国获得奥运奖牌的比例不断缩小、穷国获得奖牌总数的比例不断扩大的趋势还会在本届奥运会上继续下去。 
	
伯纳德说,他这一发现和全球经济的发展趋势相符合,一些发展中国家的生活水平得到改善,从而也扩大了这些国家分享奥运光荣的机会。中国的情况最能说明两位经济学家的经济分析模式。随着中国经济起飞,中国在奥运会上获得的奖牌数量从1988年的二十八枚猛增到2000年的五十九枚。
	
\textbf{伯纳德和布斯预测中国将在本届奥运会上获得五十七枚奖牌,但是伯纳德表示,中国的奖牌总数很可能超出他们的预测,因为中国将举办2008年奥运会,未来东道主的效应可能会使中国运动员超常发挥。}
	
\subsection{最后结果}
\begin{longtable}{@{\extracolsep{1em}}llllll}
	\caption{20004年雅典奥运会奖牌榜}\\
	\hline
	国家      & (金) & (银) & (铜) & 最后结果 & 经济学家的预测结果 \\
	\hline
	1 美国    & 35  & 39  & 29  & 103  & 93        \\
	2 中国    & 32  & 17  & 14  & 63   & 57        \\
	3 俄罗斯   & 27  & 27  & 38  & 92   & 83        \\
	4 澳大利亚  & 17  & 16  & 16  & 49   & 54        \\
	5 日本    & 16  & 9   & 12  & 37   &           \\
	6 德国    & 14  & 16  & 18  & 48   & 55        \\
	7 法国    & 11  & 9   & 13  & 33   & 33        \\
	8 意大利   & 10  & 11  & 11  & 32   & 37        \\
	9 韩国    & 9   & 12  & 9   & 30   &           \\
	10 英国   & 9   & 9   & 12  & 30   &           \\
	11 古巴   & 9   & 7   & 11  & 27   &           \\
	12 乌克兰  & 9   & 5   & 9   & 23   &           \\
	13 匈牙利  & 8   & 6   & 3   & 17   &           \\
	14 罗马尼亚 & 8   & 5   & 6   & 19   &           \\
	15 希腊   & 6   & 6   & 4   & 16   &            \\
	\hline
\end{longtable}

\section{学生邮件}

\noindent 尊敬的蒋岳祥老师:

您好!我叫是2003级博士生,曾上过您的计量课。我现在遇到一个计量方面的问题,希望能得到您的指点。

最近我在研究城市化与农业产值的关系时,通过构建模型得到如下的公式:
\begin{align}
	x_{1}=B_{1}\left(\rho-z_{u}\right)^{\alpha} z_{u}^{1-\alpha}
\end{align}	
其中$ x_{1} $为人均农业产值,而$ z_{u} $代表城市化率,$ B_{1} $,$ \rho $,$ \alpha $均为参数。

为简化起见,令$ \alpha = 0.5 $,且我们用的是1978年-2003年26年的时间序列数据,拟用扣除时间趋势的OLS方法进估计(Wooldridge,中文版,2003,pp.318-325),因此得到公式如下:
\begin{align}
	x_{1}^{2}=\beta_{0}+\beta_{1} z_{u}+\beta_{2} z_{u}^{2}+\beta_{3} t+\beta_{4} t^{2}+\mu_{t}
\end{align}
数据\footnotemark[1]如下:

\begin{table}[ht!]
	\centering
	\caption{城市化与农业产值}
	\begin{tabular}{cccccc}
		\toprule
		年份   & 人均农业产值(元)$x_{1}$ & 城市化率(%)$z_{u}$ & 年份   & 人均农业产值(元)$x_{1}$ & 人均农业产值(元)$z_{u}$ \\
		\midrule
		1978 & 116.09    & 0.18    & 1991 & 207.92    & 0.27      \\
		1979 & 133.21    & 0.19    & 1992 & 211.77    & 0.27      \\
		1980 & 136.28    & 0.19    & 1993 & 218.64    & 0.28      \\
		1981 & 147.67    & 0.20    & 1994 & 246.63    & 0.29      \\
		1982 & 162.67    & 0.21    & 1995 & 275.55    & 0.29      \\
		1983 & 175.89    & 0.22    & 1996 & 292.82    & 0.30      \\
		1984 & 193.78    & 0.23    & 1997 & 294.25    & 0.32      \\
		1985 & 184.84    & 0.24    & 1998 & 307.77    & 0.33      \\
		1986 & 189.86    & 0.25    & 1999 & 311.69    & 0.35      \\
		1987 & 198.46    & 0.25    & 2000 & 308.87    & 0.36      \\
		1988 & 191.24    & 0.26    & 2001 & 322.30    & 0.38      \\
		1989 & 178.88    & 0.26    & 2002 & 334.99    & 0.39      \\
		1990 & 208.63    & 0.26    & 2003 & 331.90    & 0.41      \\
		\bottomrule  
	\end{tabular}
\end{table}

\footnotetext[1]{数据来源:中国统计年鉴2001年、2004年,其中人均农业产值为实际值}

经运算后,得到结果如下:
	\begin{longtable}{@{\extracolsep{3em}}lcc}
		\caption{人均农业产值与城市化的关系}  \\
		\hline
		\multicolumn{3}{r}{人均农业产值平方$ x_{1}^{2} $} \\                                                                                                     
		& (1)     & (2)   \\
		\hline                                                    
		城市化率$ z_{u} $                                            & \begin{tabular}[c]{@{}c@{}}-641046\\ (-1.33)\end{tabular} & \begin{tabular}[c]{@{}c@{}}2685430\\ (3.63)\end{tabular}    \\
		城市化率平方$ z_{u}^{2} $                                           & \begin{tabular}[c]{@{}c@{}}1162719\\ (2.05)\end{tabular}  & \begin{tabular}[c]{@{}c@{}}-4241057\\ (-3.71)\end{tabular}  \\
		一次时间趋势$ t $                                         & \begin{tabular}[c]{@{}c@{}}3834.87\\ (2.47)\end{tabular}  & \begin{tabular}[c]{@{}c@{}}-9771.19\\ (-3.36)\end{tabular}  \\
		二次时间趋势$ t^{2} $ &    & 

		\begin{tabular}[c]{@{}c@{}}
			429.47\\ (5.04)
		\end{tabular}     \\
		常数项Constant                                     & 
		\begin{tabular}[c]{@{}c@{}}
			87623.44\\ (1.24)
		\end{tabular} & 
		\begin{tabular}[c]{@{}c@{}}
			-316236.2\\ (-3.37)
		\end{tabular} \\
		$ R^{2} $                       & 0.9466                                                    & 0.9758                                                      \\
		$ \bar{R}^{2} $ & 0.9393                                   & 0.9712                                                      \\
		$F$                                               & 130.03                                                    & 211.87                                                      \\
		样本数Observations                                 & 26                                                        & 26                                                           \\
		\hline                                
	\end{longtable}

由上可知,在扣除了二次时间趋势后,我们得到了比较满意的结果。

请问,这种估计方法用在这里合适吗?是否要用到非线性计量分析的方法?如何进一步改进?谢谢您的指导。

此致

\noindent 敬礼!

\hspace{10cm} 于浙大玉泉

\hspace{10cm} 2004年11月22日

\newpage
\section{2005年计量经济学作业参考答案}

\subsection{2005年计量经济学作业参考答案一}

 1. 用$ X $解释$ Y $,其中$ X,Y $~二元正态分布函数,求出回归方程	
\begin{equation*}
	\begin{aligned}
		& \because(X, Y) \sim N\left(\mu_{1}, \mu_{2}, \rho, \sigma_{1}^{2}, \sigma_{2}^{2}\right) \\
		& \therefore f(x, y)=\frac{1}{2 \pi \sigma_{1} \sigma_{2} \sqrt{1-\rho^{2}}} \exp \left\{-\frac{1}{2\left(1-\rho^{2}\right)} \cdot\left[\frac{\left(x-\mu_{1}\right)^{2}}{\sigma_{1}^{2}}-\frac{2 \rho\left(x-\mu_{1}\right)\left(y-\mu_{2}\right)}{\sigma_{1} \sigma_{2}}+\frac{\left(y-\mu_{2}\right)^{2}}{\sigma_{2}^{2}}\right]\right\} \\
		& \text { 则 } f_{Y \mid X=x}(y \mid x) = \frac{f(x, y)}{f_{x}(x)} 
		= \frac{1}{2 \pi \sigma_{2} \sqrt{1-\rho^{2}}} \exp \left\{-\frac{1}{2 \sigma_{2}^{2}\left(1-\rho^{2}\right)} \cdot\left[y-\mu_{2}-\rho \frac{\sigma_{2}}{\sigma_{1}} \cdot\left(x-\mu_{1}\right)\right]^{2}\right\} \\
		& \sim N\left(\mu_{2}-\rho \frac{\sigma_{2}}{\sigma_{1}} \cdot\left(x-\mu_{1}\right), \sigma_{2}^{2}\left(1-\rho^{2}\right)\right) \\
		& \text { 故 } \mathbb{E}(Y \mid X=x)=\mu_{2}-\rho \frac{\sigma_{2}}{\sigma_{1}} \cdot\left(x-\mu_{1}\right) \\
		& \text { 即 }\mathbb{E}(Y \mid X)=\mu_{2}-\rho \frac{\sigma_{2}}{\sigma_{1}} \cdot\left(x-\mu_{1}\right) \text { 为其回归方程 }
	\end{aligned}
\end{equation*}

 2. In the country of Wrknam, the velocity of money is constant. Real GDP grows by 5\% per year, the money stock grows by 14\% per year, and the norminal interest rate is 11 percent. What is the real interest rate?

 Solution:
 
 $ V=\dfrac{\text { Norminal GDP }}{M} \text{ by definition }  $

 $ \because \dfrac{\Delta M}{M}=14\%, \dfrac{\Delta V}{V} \text{ is constant } $ \vspace{0.5em}

 $ \therefore \dfrac{\Delta \text { Norminal } \operatorname{GDP}}{\text { Norminal } \mathrm{GDP}}=\dfrac{\Delta M}{M}=14 \%  $ \vspace{0.5em}

 $ \text{Thus, inflation rate} = \text{Norminal GDP}- \text{Real GDP}=14\%-5\%=9\%  $

 $ \text{Then, real interest rate} = 11\%-9\%=2\% $

\newpage

\subsection{2005年计量经济学作业参考答案二}	

\noindent 1. 构造$ \mu_{1}=\rho \mu_{2} $的估计量进行检验。其中$ X_{1}, X_{2} \cdots X_{m} $来自正态总体$ X \sim N\left(\mu_{1}, \sigma^{2}\right) $,$ Y_{1}, Y_{2} \cdots Y_{n} $来自正态总体$ Y \sim N\left(\mu_{1}, \sigma^{2}\right) $,$ X $、$ Y $独立,均值$ \bar{X} $,$ \bar{Y} $。
\begin{flalign*}
	&\text { 解 }: X \sim N\left(\mu_{1}, \sigma^{2}\right), Y \sim N\left(\mu_{2}, \sigma^{2}\right) & \\
	&\bar{X}-\rho \bar{Y} \sim N\left(\mu_{1}-\rho \mu_{2}, \frac{\sigma^{2}}{m}+\frac{\sigma^{2}}{n} \rho^{2}\right) & \\
	&\therefore \frac{(\bar{X}-\rho \bar{Y})-\left(\mu_{1}-\rho \mu_{2}\right)}{\sqrt{\frac{\sigma^{2}}{m}+\frac{\sigma^{2}}{n} \rho^{2}}} \sim N(0,1) & \\
	&\text { 同时,我们有 } \frac{m s_{1}^{2}}{\sigma^{2}} \sim \chi^{2}(m-1), \frac{n s_{2}^{2}}{\sigma^{2}} \sim \chi^{2}(n-1) & \\
	&\text { 令 } M=\frac{m s_{1}^{2}}{\sigma^{2}}+\frac{n s_{2}^{2}}{\sigma^{2}} \sim \chi^{2}(m+n-2) & \\
	&\text { 整理有 } \frac{(\bar{X}-\rho \bar{Y})-\left(\mu_{1}-\rho \mu_{2}\right)}{\sqrt{m s_{1}^{2}+n s_{2}^{2}}} \sqrt{(m+n-2)\left(\frac{m n}{m+n \rho^{2}}\right)} \sim t(m+n-2) &
\end{flalign*}

\noindent 2. 假设对于同一个参数$ \theta $,你有n个相互独立的无偏估计量$ \hat{\theta}_{1},\cdots ,\hat{\theta}_{n} $,它们的方差分别为$ v_{1},\cdots ,v_{n} $。那么什么样的线性组合$ \hat{\theta}=c_{1} \hat{\theta}_{1}+\cdots+c_{n} \hat{\theta}_{n} $是$ \theta $的最小方差无偏估计量?
\begin{flalign*}
	&\text { 解: }\mathbb{E}(\hat{\theta})=\left(\sum_{i=1}^{n} c_{i}\right) \theta=\theta & \\
	&\therefore \sum_{i=1}^{n} c_{i}=1 \cdots \cdots(1) & \\
	&\text{ 同时 } \operatorname{Var}(\hat{\theta})=\sum_{i=1}^{n} c_{i}^{2} v_{i} \cdots \cdots(2) & \\
	&\text{ 问题转化为在条件(1)下,求(2)的最小值} & \\
	&\text{利用拉格朗日函数,求解} & \\
	&L\left(c_{1} \cdots c_{n}, \lambda\right)=\sum_{i=1}^{n} c_{i}^{2} v_{i}-2 \lambda \sum_{i=1}^{n}\left(c_{i}-1\right) & \\
	&\frac{\partial L\left(c_{1} \cdots c_{n,} \lambda\right)}{\partial c_{i}}=2 v_{i} c_{i}-2 \lambda=0 \cdots \cdots(3) & \\
	&\frac{\partial L\left(c_{1} \cdots c_{n,} \lambda\right)}{\partial \lambda}=\sum_{i=1}^{n} c_{i}-1=0 \cdots \cdots(4) & \\
	&\text{由(3)得:}c_{i}=\frac{\lambda}{v_{i}}, \text{代入(4)} \rightarrow \lambda=\frac{1}{\sum_{i=1}^{n} v_{i}^{-1}},
	c_{i}=\frac{1}{v_{i} \sum_{i=1}^{n} v_{i}^{-1}} & \\
	&\text{从而}\operatorname{Var}(\hat{\theta})=\sum_{i=1}^{n} c_{i}^{2} v_{i}=\frac{1}{\left(\sum_{i=1}^{n} v_{i}^{-1}\right)^{2}} \sum \frac{1}{v_{i}^{2}} v_{i}=\frac{1}{\sum_{i=1}^{n} v_{i}^{-1}} \leq \frac{1}{\frac{1}{v_{i}}}=v_{i} & \\
	&\text{即说明} \hat{\theta} \text{的方差比任何一个估计量} \hat{\theta}_{i} \text{的方差小} &
\end{flalign*}

\subsection{2005年计量经济学作业参考答案三}
 1. 验证$ \text{SSE} / \sigma^{2} \sim \chi^{2} \left ( n-2 \right ) $不需要$ \beta = 0 $的假设条件即成立。
\begin{myproof}
	$$ \frac{\text{SSE}}{\sigma^{2}}=\left(\frac{Y}{\sigma}\right)^{\prime}\left(M_{0}-S_{x x} C C^{\prime}\right)\left(\frac{Y}{\sigma}\right) $$
\end{myproof}

可以验证 $ S_{x x} C C^{\prime} $  是幕等矩阵:
$$  S_{x x} C C^{\prime}S_{x x} C C^{\prime}=S_{x x}^{2} C\left(C^{\prime} C\right) C^{\prime}=S_{x x} C C^{\prime} $$

 且 $ r\left(S_{x x} C C^{\prime}\right)=\operatorname{tr}\left(S_{x x} C C^{\prime}\right)=S_{x x} \sum_{i} C_{i}^{2}=1 $
 易验证 $ M_{0}-S_{x x} C C^{\prime} $ 也是幂等矩阵
 \begin{align*}
	\left(M_{0}-S_{x x} C C^{\prime}\right)^{2}& = M_{0}-S_{x x} C C^{\prime} M_{0}-S_{x x} M_{0} C C^{\prime}+S_{x x} C C^{\prime}  \\
	&=M_{0}-S_{x x} C C^{\prime}+\frac{1}{n} S_{x x} C C^{\prime} i i^{\prime}+\frac{1}{n} S_{x x} i i^{\prime} C C^{\prime} 
	=M_{0}-S_{x x} C C^{\prime}
 \end{align*}

 且 $ r\left(M_{0}-S_{x x} C C^{\prime}\right)=\operatorname{tr}\left(M_{0}-S_{x x} C C^{\prime}\right)=n-2  $,所以,
 $ \frac{\text{SSE}}{\sigma^{2}} = \left(\frac{\alpha i+\beta x+\varepsilon i}{\sigma}\right)^{\prime}
     \left(M_{0}-S_{x x} C C^{\prime}\right)\left(\frac{\alpha i+\beta x+\varepsilon i}{\sigma}\right) $

 又
\begin{align*}
	 \left(M_{0}-S_{x x} C C^{\prime}\right)(\alpha i+\beta x) & = \alpha M_{0} i-\alpha S_{x x} C C^{\prime} i+\beta M_{0} x-\beta S_{x x} C C^{\prime} x \\
	& = \alpha\left(I-\frac{1}{n} i i^{\prime}\right) i-\alpha S_{x x} C C^{\prime} i+\beta\left(I-\frac{1}{n} i i^{\prime}\right) x-\beta S_{x x} C C^{\prime} x 
\end{align*}

$ \because \frac{1}{n} i i^{\prime} i=i, C^{\prime} i=0  , \therefore \alpha\left(I-\frac{1}{n} i i^{\prime}\right) i=\alpha S_{x x} C C^{\prime} i=0$
\[ \beta\left(I-\frac{1}{n} i i^{\prime}\right) x=\beta\left(x-\frac{1}{n} i i^{\prime} x\right)
     =\beta\left(x-\frac{1}{n} i n \bar{x}\right)=\beta(x-i \bar{x}) \]
\[ C^{\prime} x=\frac{\sum\left(x_{i}-\bar{x}\right) x_{i}}{S_{x x}}
      =\frac{\sum\left(x_{i}-\bar{x}\right)\left(x_{i}-\bar{x}\right)}{S_{x x}}=1 \]

$ \therefore \beta S_{x x} C C^{\prime} x=\beta S_{x x} C=\beta(x-i \bar{x})$

$ \therefore\left(M_{0}-S_{x x} C C^{\prime}\right)(\alpha i+\beta x)=0 $
\[ \left(\frac{Y}{\sigma}\right)^{\prime}\left(M_{0}-S_{x x} C C^{\prime}\right)\left(\frac{Y}{\sigma}\right)
    =\left(\frac{\varepsilon}{\sigma}\right)^{\prime}\left(M_{0}-S_{x x} C C^{\prime}\right)\left(\frac{\varepsilon}{\sigma}\right) \]

	故 $ \frac{\text{SSE}}{\sigma^{2}} \sim \chi^{2}(n-2) $   且与  $ \beta=0 $  无关 
\subsection{2005年计量经济学作业参考答案四}	
  1. 对于线性统计模型$ \boldsymbol{y} = \boldsymbol{X\beta}+\boldsymbol{\varepsilon} $,假设$ \boldsymbol{\varepsilon}\sim N\left(\boldsymbol{0}, \sigma^{2} \boldsymbol{I_{n}}\right), n=13, k=3 $,最小化误差平方和$ (\boldsymbol{y}-\boldsymbol{X \beta})^{\prime}(\boldsymbol{y}-\boldsymbol{X \beta}) $得到如下线性方程组:
\begin{equation}
	\left\{\begin{aligned}
		b_{1}+2 b_{2}+b_{3}=3 \\
		2 b_{1}+5 b_{2}+b_{3}=9 \\
		b_{1}+b_{2}+6 b_{3}=-8 \nonumber
	\end{aligned}\right.
\end{equation}

 (1) 把这个方程组写成矩阵的形式,并利用矩阵方法求最小二乘估计量$ \boldsymbol{b} $的值。
 $$ \left(\begin{array}{lll}
	1 & 2 & 1 \\
	2 & 5 & 1 \\
	1 & 1 & 6
\end{array}\right)\left(\begin{array}{l}
	b_{1} \\
	b_{2} \\
	b_{3}
\end{array}\right)=\left(\begin{array}{c}
	3 \\
	9 \\
	-8
\end{array}\right) \Rightarrow\left(\begin{array}{l}
	b_{1} \\
	b_{2} \\
	b_{3}
\end{array}\right)=\left(\begin{array}{c}
	3 \\
	1 \\
	-2
\end{array}\right) $$

 (2) 如果$ \boldsymbol{y}^{\prime} \boldsymbol{y}=53 $,求$ \sigma^{2} $的无偏估计量$ s^{2} $的值。
$$  \sigma^{2} \text{ 的无偏估计量 } ~  ~
 s^{2}= \frac{\boldsymbol{e}^{\prime} \boldsymbol{e}}{n-k}=\frac{\boldsymbol{y}^{\prime} 
  \boldsymbol{y}-2 \boldsymbol{\beta}^{\prime} \boldsymbol{X}^{\prime} \boldsymbol{y}+\boldsymbol{\beta}^{\prime} 
  \boldsymbol{X}^{\prime} \boldsymbol{X} \boldsymbol{\beta}}{n-k} = \frac{19}{10}=1.9  $$

 (3) 求$ \boldsymbol{b} $的协方差矩阵。
 $$ [\boldsymbol{b}]=s^{2}\left(\boldsymbol{X}^{\prime} \boldsymbol{X}\right)^{-1}=\frac{19}{10}
 	\left[\begin{array}{ccc}
	\frac{29}{4} & \frac{-11}{4} & \frac{-3}{4} \\
	\frac{-11}{4} & \frac{5}{4} & \frac{1}{4} \\
	\frac{-3}{4} & \frac{1}{4} & \frac{1}{4}
	\end{array}\right] $$

 (4) 分别写出能够检验$ H_{0}: \beta_{k}=\beta_{k}^{0} $的$ t $统计量($ k=1,2,3 $)。
 \begin{align*}
    t_{1} & = \frac{b_{1}-\beta_{1}^{0}}{\sqrt{s^{2} s_{11}}}=\frac{3-\beta_{1}^{0}}{\sqrt{1.9 \times \frac{29}{4}}} \sim t(10) \\
	t_{2} & = \frac{b_{2}-\beta_{2}^{0}}{\sqrt{s^{2} s_{22}}}=\frac{1-\beta_{2}^{0}}{\sqrt{1.9 \times \frac{5}{4}}} \sim t(10)  \\
	t_{3} &=  \frac{b_{3}-\beta_{3}^{0}}{\sqrt{s^{2} s_{33}}}=\frac{-2-\beta_{3}^{0}}{\sqrt{1.9 \times \frac{1}{4}}} \sim t(10)  
 \end{align*}

 (5) 写出能够检验$ H_{0}: \beta_{1}+\beta_{2}-2 \beta_{3}=q $的$ t $统计量和$ F $统计量。
 记   $ \boldsymbol{R b}=q, \boldsymbol{R}=\left(1 \quad 1 \quad -2 \right) $ 
 \begin{align*} 
	F &= \frac{(\boldsymbol{Rb}-q)^{\prime}\left[\boldsymbol{R}\left(\boldsymbol{X}^{\prime} \boldsymbol{X}\right)^{-1} 
	      \boldsymbol{R}^{\prime}\right]^{-1}(\boldsymbol{Rb}-q)}{\mathrm{s}^{2}J}, J=1 & \\
	  &  =\frac{(8-\mathrm{q})^{\prime}\frac{1}{6}(8-\mathrm{q})}{1.9}=\frac{(8-\mathrm{q})^{2}}{11.4} \sim F(1,10) & \\
	t & = \sqrt{F}=\frac{(8-q)}{\sqrt{11.4}} \sim t(10) 
 \end{align*}

\subsection{总结} 
$$ \text{模型}=\left\{
\begin{aligned}
	&\text{非线性回归模型} \\
	&\text{线性回归模型}{ \left\{
		\begin{aligned}
			\text{普通最小二乘估计(OLS)} \\
			\text{最大似然估计(MLE)}
		\end{aligned}\right.}
\end{aligned}\right. $$
\textbf{最小二乘法的有限样本特性}

古典回归模型的基本假设是:

\uppercase\expandafter{\romannumeral1}. $ \boldsymbol{y} = \boldsymbol{X\beta}+\boldsymbol{\varepsilon} $

\uppercase\expandafter{\romannumeral2}. $ X $是秩为$ K $的$ n \times K $非随机矩阵。

\uppercase\expandafter{\romannumeral3}. $ \mathbb{E}\left [ \boldsymbol{\varepsilon} \right ] = \boldsymbol{0} $。

\uppercase\expandafter{\romannumeral4}. $ \mathbb{E}\left[\boldsymbol{\varepsilon} \boldsymbol{\varepsilon}^{\prime}\right]=\sigma^{2}\boldsymbol{I} $。

\uppercase\expandafter{\romannumeral5}. $ \boldsymbol{\varepsilon} \sim N\left(\boldsymbol{0}, \sigma^{2} \boldsymbol{I}\right) $

未知参数$ \boldsymbol{\beta} $和$ \sigma^{2} $的最小二乘估计量是
$$ \boldsymbol{b}=\left(\boldsymbol{X}^{\prime} \boldsymbol{X}\right)^{-1} \boldsymbol{X}^{\prime} \boldsymbol{y},\quad s^{2}=\frac{\boldsymbol{e}^{\prime} \boldsymbol{e}}{(n-K)} $$
通过分析
$$ \boldsymbol{b}=\boldsymbol{\beta}+\left(\boldsymbol{X}^{\prime} \boldsymbol{X}\right)^{-1} \boldsymbol{X}^{\prime} \boldsymbol{\varepsilon} $$
并且
$$ s^{2}=\frac{\boldsymbol{\varepsilon}^{\prime} \boldsymbol{M} \boldsymbol{\varepsilon}}{n-K} $$

我们可得下列精确的有限样本结果:

1. $ \mathbb{E}\left [ \boldsymbol{b} \right ] = \boldsymbol{\beta} $(最小二乘估计是无偏的)
	
2. $ \operatorname{Var}[\boldsymbol{b}]=\sigma^{2}\left(\boldsymbol{X}^{\prime} \boldsymbol{X}\right)^{-1} $
	
3. 任意函数$ \boldsymbol{r}^{\prime} \boldsymbol{\beta} $的最小方差线性无偏估计量是$ \boldsymbol{r}^{\prime} \boldsymbol{b} $。(这就是高斯—马尔科夫定理)	
	
4. $ \mathbb{E}\left[s^{2}\right]=\sigma^{2} $

5. $ \text { Cov }[\boldsymbol{b},\boldsymbol{e}] = \boldsymbol{0} $

为了构造置信区间和检验假设,我们根据正态分布的假设\uppercase\expandafter{\romannumeral5}推导额外了的结果,即

6. $ \boldsymbol{b} $和$ \boldsymbol{e} $在统计上是相互独立的。相应的,$ \boldsymbol{b} $和$ s^{2} $无关并在统计上相互独立。

7. $ \boldsymbol{b} $的精确分布依赖于$ \boldsymbol{X} $,是$ N\left(\boldsymbol{\beta}, \sigma^{2}\left(\boldsymbol{X}^{\prime} \boldsymbol{X}\right)^{-1}\right) $。

8. $ (n-K) s^{2} / \sigma^{2} $的分布是$ \chi^{2}(n-K) $。$ s^{2} $的均值是$ \sigma^{2} $,方差是$ 2 \sigma^{4} /(n-K) $。

9. 根据6至8结果,统计量$ t(n-K)=\frac{b_{k}-\beta_{k}}{s^{2}\left(X^{\prime} X\right)_{k k}^{-1}} $服从自由度为$ n-K $的$ t $分布。
	
10. 用于检验一组$ J $个线性约束$ \boldsymbol{R\beta} = q $的检验统计量	
$$ \frac{(\boldsymbol{Rb}-q)^{\prime}\left[\boldsymbol{R}\left(\boldsymbol{X}^{\prime} \boldsymbol{X}\right)^{-1} \boldsymbol{R}^{\prime}\right]^{-1}(\boldsymbol{R b}-q) / J}{\boldsymbol{e}^{\prime} \boldsymbol{e} /(n-K)}=\frac{(\boldsymbol{Rb}-q)^{\prime}\left[\boldsymbol{R} s^{2}\left(\boldsymbol{X}^{\prime} \boldsymbol{X}\right)^{-1} \boldsymbol{R}^{\prime}\right]^{-1}(\boldsymbol{Rb}-q)}{J} $$
服从自由度为$ J $和$ n-K $的$ F $分布。	

注意,利用\uppercase\expandafter{\romannumeral1}至\uppercase\expandafter{\romannumeral5}建立起来的$ \boldsymbol{b} $的各种性质和根据扰动项更进一步的正态分布假设而得到的额外推断结果之间的区别。第一组中最重要的结果是高斯—马尔科夫定理,\textbf{它与扰动项的分布无关}。根据正态分布假设得到的重要的附加结果是7、8、9、10。\textbf{正态性没有产生任何额外的有限样本的最优性结果}。
