 %%%%%%%%%%%%%%%%%%%%%%%%%%%%%%%%%%%%%%%%%% 
 % @File    : c:\Users\Administrator\Desktop\Econometrics\sections\NL.tex
 % @Date    : 2021-01-22 13:51:14
 % @Author  : RankFan
 % @Email   : 1917703489@qq.com
 % -----
 % Last Modified: 2021-02-15 14:21:38
 % Modified By: Rank_fan
 % -----
 %%%%%%%%%%%%%%%%%%%%%%%%%%%%%%%%%%%%%%%%%% 

 \chapter{Preface}

各位读者好,我现在是一名经管学科的在读研究生,在大三时看过浙江大学蒋岳祥老师的中级计量经济学,后来又看了部分高级计量经济学。
推荐资料:\href{https://bbs.pinggu.org/a-503441.html}{Econometrics Analysis} ;推荐课程:
蒋岳祥老师的\href{https://www.bilibili.com/video/BV1es411W7KU?from=search&seid=3253765381861883959}{中高级计量经济学},
厦门大学洪永淼老师的
\href{https://www.icourse163.org/course/XMU-1002606048}{高级计量经济学}  和
\href{https://www.bilibili.com/video/BV11t411A7bp}{概率论与统计学}

\vspace{0.5em}
我被蒋岳祥老师对于计量经济学所吸引了,蒋岳祥老师课程曾将“小概率”事件与佛学的“因果循环”联系在一起,顿时让我感受到计量经济学的“哲学”性质。

\vspace{0.5em}
后来我在网上搜集到蒋岳祥老师的讲稿,由于计量经济学课程公式很多,\LaTeX 在这方面又有很大的优势,历时半年多,我终于将蒋岳祥老师讲稿的主要内容整理
成这个总结。

\vspace{0.5em}
这个总结主要分为四个部分,第一部分:“基础” ;第二部分:“线性”;第三部分:“应用” ;第四部分:“附录”。

\vspace{0.5em}
非常感谢我的同学给予的支持,由于我的能力有限,在整理蒋岳祥老师讲稿时未免不出现一些错误,请各位读者见谅。


~ \ \
\\[3em]

\begin{flushright}
    RankFan \quad  \\ 
    \today
\end{flushright}


% \hfill {2021年2月8日}